\documentclass[aspectratio=169]{beamer}

\mode<presentation>
\usetheme{Boadilla}
\definecolor{Khaki}{RGB}{144,110,62}
\definecolor{blue}{RGB}{30,90,205}
\definecolor{red}{RGB}{213,94,0}
\definecolor{green}{RGB}{0,128,0}
\setbeamercolor{title}{fg=Khaki}
\setbeamercolor{frametitle}{fg=Khaki}
\setbeamercolor{block title}{bg=Khaki, fg=white}
\setbeamercolor{block body}{bg=white}
\setbeamercolor{structure}{fg=Khaki}
\setbeamercolor{item projected}{fg=white}
\setbeamercolor{item}{fg=Khaki}
\setbeamercolor{subitem}{fg=Khaki}
\setbeamercolor{section in toc}{fg=Khaki}
\setbeamercolor{description item}{fg=Khaki}
\setbeamercolor{caption name}{fg=Khaki}
\setbeamercolor{button}{bg=Khaki, fg=white}
\setbeamercolor{caption name}{fg=Khaki}
\usepackage{graphics}
\usepackage{tikz}
\usepackage{amsmath}
\usepackage{bbm}
\usetikzlibrary{decorations.pathreplacing}
\usepackage{geometry}
\usepackage{booktabs}
\usepackage{bookmark}
\usepackage{multirow, makecell}
\usepackage{float}
\usepackage{fancyvrb}
\usepackage{caption}
\usepackage{subcaption}
\usepackage{adjustbox}
\usepackage{hyperref}
\usepackage{threeparttable}
\usepackage{appendixnumberbeamer} 
\usepackage[T1]{fontenc}
\usepackage[default]{lato} 

\newenvironment{wideitemize}{\itemize\addtolength{\itemsep}{10pt}}{\enditemize}
\newenvironment{wideenumerate}{\enumerate\addtolength{\itemsep}{10pt}}{\endenumerate}
\newenvironment{widedescription}{\description\addtolength{\itemsep}{10pt}}{\enddescription}
\hypersetup{
colorlinks=true,
linkcolor=black,
filecolor=green, 
urlcolor=blue,
}
\beamertemplatenavigationsymbolsempty
\setbeamercolor{author in head/foot}{bg=white, fg=Khaki}
\setbeamercolor{title in head/foot}{bg=white, fg=Khaki}
\setbeamercolor{date in head/foot}{bg=white, fg=Khaki}
\setbeamercolor{section in head/foot}{bg=white, fg=Khaki}
\setbeamercolor{page number in head/foot}{bg=white, fg=Khaki}
\setbeamercolor{headline}{bg=white}
\setbeamertemplate{footline}{
    \leavevmode%
    \hbox{%
        \begin{beamercolorbox}[wd=.33\paperwidth,ht=2.25ex,dp=1ex,center]{date in head/foot}%
            \usebeamerfont{date in head/foot}\insertshortdate
        \end{beamercolorbox}%
        \begin{beamercolorbox}[wd=.44\paperwidth,ht=2.25ex,dp=1ex,center]{title in head/foot}%
            \usebeamerfont{title in head/foot}\insertshorttitle
        \end{beamercolorbox}%
        \begin{beamercolorbox}[wd=.22\paperwidth,ht=2.25ex,dp=1ex,center]{page number in head/foot}%
            \usebeamerfont{page number in head/foot}\insertframenumber{} / \inserttotalframenumber
        \end{beamercolorbox}}%
        \vskip0pt%
    }
%\setbeamercolor{page number in head/foot}{fg=black}
\setbeamertemplate{section in toc}[sections numbered]
\setbeamertemplate{subsection in toc}{\leavevmode\leftskip=3em\rlap{\hskip-1.75em\inserttocsectionnumber.\inserttocsubsectionnumber}\inserttocsubsection\par}
\setbeamerfont{subsection in toc}{size=\footnotesize}

\newenvironment{transitionframe}{\setbeamercolor{background canvas}{bg=Khaki}\setbeamertemplate{footline}{} \begin{frame}}{\end{frame}}


\makeatletter
\let\@@magyar@captionfix\relax
\makeatother


\title[Recitation 5 (UN 3412)]{Recitation 5: Extended specifications and assessing the models} % Change this regularly
\author[Lee]{Seung-hun Lee}
\institute[]{Columbia University \\ Undergraduate Introduction to Econometrics Recitation}

\date[October 13th, 2022]{October 13th, 2022}

\begin{document}
\begin{frame}
\titlepage
\end{frame}


%%% Color slides for section headers: Use for colloquium version (The ones bewteen \iffals and \fi)



\begin{transitionframe}
  \begin{center}
         { \Huge \textcolor{white}{Extended models: Quadratic, log, and interacted regressors}}
       \end{center}
\end{transitionframe}



\begin{frame}
\frametitle{Not everything in world is linearly related}
\begin{itemize}
\item The effect of $X$ may grow larger/smaller as $X$ increases (think of wage and age)
 \item For correlations like this, nonlinear regressors are necessary
 \item When incorporating such regressors, the interpretation of each coefficient becomes trickier.
 \item Quadratic relations: Think about wage and age - wages increase with age, but (usually) at a decreasing pace
 \[
W = \beta_0 + \beta_1 X+ \beta_2 X^2+u
\]
\item The marginal effect of $X$ on $W$ can be written as
\[
\frac{\partial W}{\partial X}=\beta_1+2\beta_2 X
\]
\end{itemize}
\end{frame}

\begin{frame}
\frametitle{Back to linear models: So how is the marginal effect different?}
\begin{itemize}
 \item In a linear regressor only format, the marginal effect is
 \[
W = \beta_0 + \beta_1 X+u
\]
\[
\frac{\partial W}{\partial X}=\beta_1
\]
\item The difference is that with quadratic terms, we can express cases where \textit{marginal} changes to $W$ with respect to $X$ is not a constant, but depends on some value of $X$
\begin{itemize}
\item In the above case, if $\beta_2>0$, marginal increase in $W$ increases with $X$ (and vice versa)
\end{itemize}
\end{itemize}
\end{frame}


\begin{frame}
\frametitle{We are interested in percent changes (because measurement units)...}
\begin{itemize}
\item We might be interested in how the changes in the $X$ in terms of \textit{percentages} affect changes in the dependent variable $Y$. 
\item (Natural) Log regressors captures the idea of percentage changes. 
\item Recall from calculus the first order approximation: For any differentiable function $f$, and a very small change in $x$, the following relationship holds. 
\[
f(x+\Delta x) \simeq f(x)+f'(x)[(x+\Delta x) -x] 
\]
\item Define $y=f(x)=\ln{x}$. Then $f(x+\Delta x) = y+\Delta y$ and $f'(x)=\frac{1}{x}$. We then get 
\[
\Delta y = \frac{\Delta x}{x}\implies \ln{(x+\Delta x)}-\ln{x} \simeq \frac{\Delta x}{x} \implies \ln\left(1+\frac{\Delta x}{x}\right)\simeq\frac{\Delta x}{x}
\]
\item Therefore, the changes in log values allows us to capture changes in percentages, at least for very small amount of change.
\end{itemize}
\end{frame}

\begin{frame}
\frametitle{How much \textit{level} changes in $Y$ due to \textit{percentage} changes in $X$?}
\begin{itemize}
\item\textbf{lin-$\log$}:  Consider the model $Y=\beta_0 + \beta_1 \log{X}+u$.
\item I take a before and after approach by changing $X$ by $\Delta x$. 
\item Then the total amount of $Y$ would be $Y+\Delta y$. Formally, 
\[
Y+\Delta y = \beta_0 + \beta_1 \log(X+\Delta x)+u
\]
Subtract $Y=\beta_0 + \beta_1 \log{X}+u$ from this equation to get
\[
\Delta y = \beta_1 \log(X+\Delta x)-\beta_1\log{X} = \beta_1 \log\left(1+\frac{\Delta x}{X} \right) = \beta_1 \frac{\Delta x}{X}
\]
\item Therefore, 
\[
\beta_1 = \frac{\Delta y}{(\Delta x/X)}
\]
Note that the percentage change in $X$ is $\frac{\Delta x}{X} \times 100$. In words, change in $X$ by 1 percent, raises $Y$ by $\beta_1 \times 0.01$. 
\end{itemize}
\end{frame}

\begin{frame}
\frametitle{How much \textit{percentage} changes in $Y$ due to \textit{level} changes in $X$?}

\begin{itemize}
\item\textbf{$\log$-lin}: Consider the model $\log{Y}=\beta_0 + \beta_1 X+u$. 
\item Conduct a similar before and after analysis as we did before to get:
\[
\log(Y+\Delta y) = \beta_0 + \beta_1(X+\Delta x)+u
\]
\item Then, subtract $\log{Y}=\beta_0 + \beta_1 X+u$. This gets us
\[
\log\left(1+\frac{\Delta y}{Y}\right)\simeq \frac{\Delta y}{Y} = \beta_1 \Delta x
\]
\item Then, $\beta_1$ can be backed out as
\[
\beta_1 = \frac{(\Delta y / Y)}{\Delta x}
\]
Again, using the fact that percentage change in $Y$ can be represented as $\frac{\Delta y}{Y}\times 100$. This implies that a 1 unit change in $X$ raises $Y$ by $(100\times \beta_1)\%$. 
\end{itemize}
\end{frame}

\begin{frame}
\frametitle{How much \textit{percentage} changes in $Y$ due to \textit{percentage} changes in $X$?}
\begin{itemize}
\item\textbf{$\log$-$\log$}: Consider the equation $\log{Y} = \beta_0 + \beta_1\log{X}+u$. 
\item Similar approach allows us to write
\[
\log(Y+\Delta y) = \beta_0 + \beta_1 \log(X+\Delta x)+u
\]
\item Subtract the original equation to obtain
\[
\log\left(1+\frac{\Delta y}{Y}\right)=\beta_1\log\left(1+\frac{\Delta x}{X}\right) \implies \frac{\Delta y}{Y}=\beta_1\frac{\Delta x}{X}
\]
\item which implies 
\[
\beta_1 = \frac{(\Delta y/Y)}{(\Delta x/X)}
\]
This implies that 1\% change in $X$ leads to $\beta_1\%$ change in $Y$. This is the \textit{elasticity} interpretation
\end{itemize}
\end{frame}

\begin{frame}
\frametitle{IRL: Marginal effect of $X_1$ on $Y$ maybe a function of some other variables!}
\begin{itemize}
\item Suppose that we are interested in the relationship between test scores ($Y$) and class size($X_1$). 
\item However, one might guess that the effect of class size may differ depending on some other variables. 
\begin{itemize}
\item e.g. schools in districts where there are more funding ($X_2$) are more likely to enjoy the benefits of small school classroom
\end{itemize}
\item In math, the marginal effect of $X_1$ on $Y$ may depend on $X_2$. 
\item To capture this idea in a model, we can incorporate an \textbf{interaction term} involving $X_1$ and $X_2$, which can be written as $X_1 \times X_2$ \par\medskip
\end{itemize}
\end{frame}

\begin{frame}
\frametitle{Binary $\times$ Binary}
\begin{itemize}
\item Suppose that there are two binary variables, $D_1, D_2$.
\item Notice the regression equation below
\[
Y= \beta_0 + \beta_1 D_1 + \beta_2 D_2 + \beta_3 (D_1 \times D_2) + u
\]  
\item By now, you know that $\beta_1$ captures the group difference in average for those in group $D_1$ and $\beta_2$ captures the same for those in group $D_2$. 
\item Note that $D_1 \times D_2$ becomes 1 only individual is in both groups. The average for these individuals are captured by $\beta_3$ coefficient. 
\item To sum up:
\begin{itemize}
\item $E[Y|D_1 = 1] = \beta_0+\beta_1+\beta_2\times D_2+\beta_3\times D_2$
\item $E[Y|D_1 = 0] = \beta_0+\beta_2\times D_2$
\item $E[Y|D_1 = 1]-E[Y|D_1 = 0] = \beta_1 + \beta_3\times D_2$
\end{itemize}
\item So the effect of $D_1$ differs depending on $D_2$ as well. 
\item This setup allows us to analyze the difference in effect of binary variable depending on another binary factor. 
\end{itemize}
\end{frame}

\begin{frame}
\frametitle{Binary $\times$ Continuous}
\begin{itemize}
\item  Instead of a second binary variable, we include a continuous variable $X_1$. 
\item Now we write
\[
Y= \beta_0 + \beta_1 D_1 + \beta_2 X_1 + \beta_3 (D_1 \times X_1) + u
\]
\item In earlier example, classroom size would be a continuous variable, so that will be our $X_1$. 
\item Now, define $D_i=1$ if a school district receives funding and $0$ if otherwise. 
\item The effect of classroom size would now be 
\[
\frac{\partial Y}{\partial X_1} = \beta_2 + \beta_3 D_1
\]
\item Now the effect of $X_1$ depends on $D_1$ - whether the district receives funding or not
\end{itemize}
\end{frame}

\begin{frame}
\frametitle{Continuous $\times$ Continuous}
Interaction terms
\begin{itemize}
\item  Consider this regression, where both $X_1$ and $X_2$ are continuous variables. 
\[
Y = \beta_0 + \beta_1X_1 + \beta_2 X_2 + \beta_3 (X_1\times X_2)+u
\] 
\item In this equation, the effect of $X_1$ on $Y$, and that for $X_2$ on $Y$ are
\[
\frac{\partial Y}{\partial X_1} = \beta_1 + \beta_3 X_2 \ \ \ \ \frac{\partial Y}{\partial X_2} = \beta_2 + \beta_3 X_1 
\]
\item Now you see that the marginal impact of $X_1$ on $Y$ is dependent of $X_2$. 
\end{itemize}
\end{frame}


\begin{transitionframe}
  \begin{center}
         { \Huge \textcolor{white}{Assessing the regression model}}
       \end{center}
\end{transitionframe}



\begin{frame}
\frametitle{Critiquing the regressions}
\begin{itemize}
\item \textbf{External validity} is concerned with the applicability of the regression model to other contexts. 
\begin{itemize}
\item For instance, you may wonder whether regression model from schools in California could explain what happens in the classrooms in South Korea
\item Any critiques to the model questioning the replicability of the  result to other dataset is assessing the external validity. 
\end{itemize}
\item \textbf{Internal validity} assesses the model from the point of view of the population being studied. 
\begin{itemize}
\item This approaches questions the validity of the statistical inference within the given dataset. There are five threats to internal validity.
\end{itemize}
\end{itemize}
\end{frame}

\begin{frame}
\frametitle{Is the model itself set properly?}
\begin{itemize}
\item \textbf{Omitted variable bias: } Did the researcher left out factors that could affect $Y$ and are correlated with $X$? 
\item \textbf{Wrong functional form: } Did the researcher incorporate $X$ in a proper functional form? (quadratic, logs?)
\item \textbf{Errors in variables bias: } Measurement error. If $X$ does not have a clear cut measure, part of $X$ correlated to $Y$ and the observed version of $X$ is left out, resulting in attenuation bias
\item \textbf{Sample selection bias: }  If certain groups are more likely to not respond, then the data fails to represent population and the estimates based on this data is biased.
\item \textbf{Simultaneous causality bias: } There are cases where $Y$ causes $X$ too, also leading to the bias (consider supply and demand)
\end{itemize}
\end{frame}

\begin{transitionframe}
  \begin{center}
         { \Huge \textcolor{white}{Midterm checklistl}}
       \end{center}
\end{transitionframe}

\begin{frame}
\frametitle{Midterm Checklist}
\begin{itemize} 
\item To make your life easier, I will organize the list of things that you need to know for the upcoming midterm exam. 
\item Ideally, you would have looked back at the key proofs and derived them yourself, run loads of regressions to be familiar with OLS formula and a lot more.
\item If you are time-constrained, you are still okay. You can look at the items below and remind yourself of what you have learned.
\item Moreover, the list here is intended to give you some guidance of what you should review in the last minute (hopefullly). 
\item The most important things (things you should know by heart by now) are listed with $\star\star\star$. The more stars there are, the more crucial they are to understand what we have learned so far. 
\item However, it does not mean that the items with fewer stars can easily be ignored. I am giving some priority weights
\end{itemize}
\end{frame}

\begin{frame}
\frametitle{Concepts you should know}
\begin{itemize} 
\item[$\star\star\star$] \textbf{Randomized control test}: This is the core design that any econometric regressions should strive for. The key idea to remember is that experimental designs should have a clear control vs treatment group structure and that the groups are otherwise identical except for treatment status. If you have this setup, an OLS regression is going to give you an accurate estimate of the (average) treatment effect. This setup is not always feasible, for which we have concepts that will be covered after the midterm exam. 
\item[$\star\star\star$] \textbf{Hypothesis test}: Some people say that this is in the core of econometrics. By now you should know how to set up a null/alternative hypothesis, what type of test distribution to use ($F$, $t$, or normal), know your test statistic, and come to the conclusion using either the test statistics, p-value, or confidence interval. Also be familiar with $R^2$ and $\bar{R}^2$
\end{itemize}
\end{frame}

\begin{frame}
\frametitle{Concepts you should know}
\begin{itemize} 
\item[$\star\star\star$] \textbf{Interpreting coefficients}: The most important thing when you get your regression output is understanding what it means. Especially in multivariate case, you should know that the $\hat{\beta}$ coefficient in front of your independent variable is the marginal effect of changing $X$ by 1 unit on $Y$, \textit{holding other variables constant}. Interpretation is very crucial for nonlinear regressors. Be familiar with the regressions that involve logs and interaction terms. 
\item[$\star\star\star$]\textbf{Omitted variable bias}: This is what motivates the use of multivariate regressions. Do remember what the two conditions are and how you can mathematically show them.
\item[$\star\star\star$] \textbf{OLS assumptions}: Not that you need to memorize them. You do need to remind yourself how you proved unbiasedness, how you showed OVB, etc. with which assumptions. 
\end{itemize}
\end{frame}

\begin{frame}
\frametitle{Concepts you should know}
\begin{itemize} 
\item[$\star\star$] \textbf{Estimating the OLS}: Star docked off since you may be given the access to the formula during the exam. However, it is essential to know where the OLS estimators originated from. 
\item[$\star\star$] \textbf{Statistical properties of OLS}: Specifically, you need to be familiar with how to prove OLS estimator is unbiased. It does not mean you should memorize every line of the proof. It helps, however, to remind yourself which assumptions were used in the proof.
\item[$\star\star$]\textbf{Heteroskedasticity}: Remember how this affects $t$-statistics. Specifically, that it alters the standard errors but not the coefficients.  
\item[$\star\star$]\textbf{Multicollinearity}: Know two types of multicollinearity and what causes them. In particular, if you are given dummy variables, understand what to do with it. Specifically, \textit{If you have $K$ dummy variables for $K$ categories, include $K-1$ variables.}
\item[$\star\star$]\textbf{Binary (independent) variables}: Know how to set them up. Also, know that the coefficient in front of it implies the mean difference with the ``benchmark'' group.
\item[$\star\star$]\textbf{Internal/External Validity}: When the question asks you to assess the regression model and results, you can use these ideas to critique the model. 
\end{itemize}
\end{frame}

\begin{frame}
\frametitle{Things to do}
\begin{itemize} 
\item[$\star\star\star$] \textbf{Do decent amount of studying, but definitely not too much}: Do get some enough sleep. It is never a good idea to cramp your econometrics knowledge overnight before the day of the exam. Always remember that slow and steady wins the race (especially for econometrics). 
\item[$\star\star\star$] \textbf{Use office hours}: I am talking about OH for both TA's and the professors. Our job is to help you get the most out of this course. We  are ready to help you in any way we can. Just visit remaining OH as often as possible and ask questions. It may just help to see what others are asking too.
\item[$\star\star\star$]\textbf{Review problem sets}: Not all of them. However, the ``pen(cil) and paper'' questions are a must. Since we assume that you are familiar with the problem set, it is important to walk back through the questions you solved before. 
\item[$\star\star\star$]\textbf{Review extra questions the professor revealed}: The questions there reflect the ideal questions that the professor wants you to know before the exam. This will also be covered on Tuesday.
\end{itemize}
\end{frame}

\begin{frame}
\frametitle{Things to do}
\begin{itemize} 
\item[$\star\star$] \textbf{Ungraded Problems in the problem set}: Some of them are empirical exercises. Nevertheless, they can also be a valuable source of new questions for those who need them. 
\item[$\star$]\textbf{End of the Chapter questions}: Since we did not cover some of the topics in the book, not all of the questions are relevant. Also note that the book approaches some matter differently from the class. Some of the questions that match what we learned, however, can be valuable for you. This is true if you need new questions to work on. 
\end{itemize}
\end{frame}

%%%%%%%%%%%
\end{document}
