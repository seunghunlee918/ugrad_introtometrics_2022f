\documentclass[aspectratio=169]{beamer}

\mode<presentation>
\usetheme{Boadilla}
\definecolor{Khaki}{RGB}{144,110,62}
\definecolor{blue}{RGB}{30,90,205}
\definecolor{red}{RGB}{213,94,0}
\definecolor{green}{RGB}{0,128,0}
\setbeamercolor{title}{fg=Khaki}
\setbeamercolor{frametitle}{fg=Khaki}
\setbeamercolor{block title}{bg=Khaki, fg=white}
\setbeamercolor{block body}{bg=white}
\setbeamercolor{structure}{fg=Khaki}
\setbeamercolor{item projected}{fg=white}
\setbeamercolor{item}{fg=Khaki}
\setbeamercolor{subitem}{fg=Khaki}
\setbeamercolor{section in toc}{fg=Khaki}
\setbeamercolor{description item}{fg=Khaki}
\setbeamercolor{caption name}{fg=Khaki}
\setbeamercolor{button}{bg=Khaki, fg=white}
\setbeamercolor{caption name}{fg=Khaki}
\usepackage{graphics}
\usepackage{tikz}
\usepackage{amsmath}
\usepackage{bbm}
\usetikzlibrary{decorations.pathreplacing}
\usepackage{geometry}
\usepackage{booktabs}
\usepackage{bookmark}
\usepackage{multirow, makecell}
\usepackage{float}
\usepackage{fancyvrb}
\usepackage{caption}
\usepackage{subcaption}
\usepackage{adjustbox}
\usepackage{hyperref}
\usepackage{threeparttable}
\usepackage{appendixnumberbeamer} 
\usepackage[T1]{fontenc}
\usepackage[default]{lato} 

\newenvironment{wideitemize}{\itemize\addtolength{\itemsep}{10pt}}{\enditemize}
\newenvironment{wideenumerate}{\enumerate\addtolength{\itemsep}{10pt}}{\endenumerate}
\newenvironment{widedescription}{\description\addtolength{\itemsep}{10pt}}{\enddescription}
\hypersetup{
colorlinks=true,
linkcolor=black,
filecolor=green, 
urlcolor=blue,
}
\beamertemplatenavigationsymbolsempty
\setbeamercolor{author in head/foot}{bg=white, fg=Khaki}
\setbeamercolor{title in head/foot}{bg=white, fg=Khaki}
\setbeamercolor{date in head/foot}{bg=white, fg=Khaki}
\setbeamercolor{section in head/foot}{bg=white, fg=Khaki}
\setbeamercolor{page number in head/foot}{bg=white, fg=Khaki}
\setbeamercolor{headline}{bg=white}
\setbeamertemplate{footline}{
    \leavevmode%
    \hbox{%
        \begin{beamercolorbox}[wd=.33\paperwidth,ht=2.25ex,dp=1ex,center]{date in head/foot}%
            \usebeamerfont{date in head/foot}\insertshortdate
        \end{beamercolorbox}%
        \begin{beamercolorbox}[wd=.44\paperwidth,ht=2.25ex,dp=1ex,center]{title in head/foot}%
            \usebeamerfont{title in head/foot}\insertshorttitle
        \end{beamercolorbox}%
        \begin{beamercolorbox}[wd=.22\paperwidth,ht=2.25ex,dp=1ex,center]{page number in head/foot}%
            \usebeamerfont{page number in head/foot}\insertframenumber{} / \inserttotalframenumber
        \end{beamercolorbox}}%
        \vskip0pt%
    }
%\setbeamercolor{page number in head/foot}{fg=black}
\setbeamertemplate{section in toc}[sections numbered]
\setbeamertemplate{subsection in toc}{\leavevmode\leftskip=3em\rlap{\hskip-1.75em\inserttocsectionnumber.\inserttocsubsectionnumber}\inserttocsubsection\par}
\setbeamerfont{subsection in toc}{size=\footnotesize}

\newenvironment{transitionframe}{\setbeamercolor{background canvas}{bg=Khaki}\setbeamertemplate{footline}{} \begin{frame}}{\end{frame}}


\makeatletter
\let\@@magyar@captionfix\relax
\makeatother


\title[Recitation 1 (UN 3412)]{Recitation 1: Understanding Stata and Randomized Control Trials} % Change this regularly
\author[Lee]{Seung-hun Lee}
\institute[]{Columbia University \\ Undergraduate Introduction to Econometrics Recitation}

\date[September 14th, 2021]{September 14th, 2021}

\begin{document}
\begin{frame}
\titlepage
\end{frame}


%%% Color slides for section headers: Use for colloquium version (The ones bewteen \iffals and \fi)


\begin{transitionframe}
  \begin{center}
         { \Huge \textcolor{white}{Logistics of the recitation}}
       \end{center}
\end{transitionframe}





\begin{frame}
\frametitle{Recitation Logistics}
\begin{wideitemize}
\item Location: 602 Northwest Corner Building
\item Time: Thursdays 1PM-2PM 
\begin{itemize}
\item 30-40 minutes will be spent on reviewing materials from the lecture, the rest will be spent on Stata demonstrations. 
\item Pending room availability, I will stay an extra 10-20 minutes to answer your questions
\item Recitation notes to be posted by noon on Thursdays for you to download.
\item Slides will be posted AFTER the recitation (before midnight on Thursdays).
\end{itemize}
\item Office Hours
\begin{itemize}
\item Zoom (\href{https://columbiauniversity.zoom.us/j/96949225512?pwd=bTgwKytIVHpmNVloU0hNOEFxQ3J3UT09}{Click here to join}) and Lehman 327 (So technically hybrid)
\end{itemize}
\item Further materials
\begin{itemize}
\item You can go \href{https://github.com/seunghunlee918/ugrad_introtometrics_2021f}{here} for my old recitation materials
\end{itemize}
 \end{wideitemize}
\end{frame}


\begin{transitionframe}
  \begin{center}
         { \Huge \textcolor{white}{Econometrics and RCT}}
       \end{center}
\end{transitionframe}

\begin{frame}
\frametitle{What is econometrics trying to achieve?}
\begin{wideitemize}
\item Econometrics is a field in economics that tries to answer real life questions.  
\begin{itemize}
\item Ultimately, it is about making a \textbf{quantitative} statement about two or more random events
\item We can make \textbf{correlational} statements, but we want to identify \textit{causal relationships}
\end{itemize}
\item In order to achieve this goal, we collect data from a suitably defined population and use various methods to estimate a parameter that implies correlational/causal relationship.
\item To fully understand what econometrics is trying to achieve, we need to ask ourselves these three questions
\begin{itemize}
\item  What is the difference between correlational and causal relation? 
\item What is the suitably defined population? 
\item What are the methods that we need to use in econometrics?
\end{itemize}
 \end{wideitemize}
\end{frame}

\begin{frame}
\frametitle{Correlation vs Causation}
\begin{wideitemize}
\item Suppose you have two random variables $X$ and $Y$. You want to identify if $X$ \textit{causes} $Y$
\item A \textbf{correlation} between $X$ and $Y$ is a statistical measure that describes how the two variables move together
\begin{itemize}
\item It captures \textit{any} type of statistical dependence that moves the two variables together: Causation, but also others too!
\item Not causal 1: $Y$ can cause $X$
\item Not causal 2: $X$ and $Y$ are jointly moving because there is $Z$ that affects both
\end{itemize}
\item A \textbf{causal} relationship: \textit{Cleanly (exogenously)} changing variables $X$ leads to changes in $Y$
\begin{itemize}
\item Much more difficult: Changes in $X$ may be a combination of many things
\item Changes from $X$ alone and changes from other factors that may indirectly affect $X$
\item RCT: Isolates clean changes in $X$ that can help us tell whether changes in $X$ affects $Y$, and by how much
\item Econometrics: We can express concisely the relationship between $X$ and $Y$ variables in a single equation. 
\end{itemize}
 \end{wideitemize}
\end{frame}

\begin{frame}
\frametitle{Suitably defined population?}
\begin{wideitemize}
\item When we say we are interested in the relationship between schooling and wages, whose effects are we interested in? 
\begin{itemize}
\item The entire US population, high school graduates, or college graduates?  
\item Determines sampling methods we use to obtain a representative and comparable sample
\item Complete randomization, stratified randomization, or cluster randomization
\end{itemize}
\item  Note: We are almost surely never going to get the data from the entire population. 
\begin{itemize}
\item Gathering data from the entire population is logistically (and maybe ethically) difficult. 
\item The estimate we are obtaining through any econometric exercise is thus a \textit{sample analogue} of the actual value we are trying to get
\item We will do diagnostic tests to see if they can be reasonably close to the true value. 
\end{itemize}
 \end{wideitemize}
\end{frame}


\begin{frame}
\frametitle{What methods?}
\begin{wideitemize}
\item In econometrics, we will use many estimation methods to obtain the sample analogue of the parameter of interest. 
\begin{itemize}
\item Ordinary least squares (OLS): Suitable for randomized control trials or in any case where the treatment assignment is as good as random.
\item Panel estimation: If data has multiple individuals and multiple time periods.
\item Instrumental variable methods: When we have proxy variables relevant to variable of interest and is reasonably exogenous
\item Difference-in-differences: When we study `before \& after' events with multiple entities
\item Regression discontinuity: In treatment with a cutoff determining treatment assignment
\item Time series: When we observe one entity over multiple periods 
\end{itemize}
\item Depending on the type of variables we use in our exercise, we have:
\begin{itemize}
\item Univariate regression: One variable (besides an overall constant) controlled for
\item Multivariate regression: Multiple variables (besides an overall constant) controlled for
\item Nonlinear regression: Binary dependent variables
\item Big data methods
\end{itemize}
 \end{wideitemize}
\end{frame}

\begin{frame}
\frametitle{Understanding RCTs}
\begin{wideitemize}
\item In \textbf{randomized control trials}, we randomly categorize some individuals under treatment group and controlled group and run various tests
\begin{itemize}
\item Benchmark for good program evaluation
\end{itemize}
\item Potential outcomes framework
\begin{itemize}
\item $Y_i$ : Observed outcome for individual $i\in\{1,...,N\}$
\item $i$: Either in treatment or control group (not both)$\to$ $W_i=1$ if $i$ is treated, 0 if otherwise
\item $\mathbf{W}$:  an $N$-tuple vector of treatment assignment for all individuals
\item Key assumption: Others' treatment assignment has no effect on my treatment (stable unit treatment value assumption (SUTVA))
\item Potential outcome $Y_i(w)$: Outcome for treated ($Y_i(1)$) and the untreated individual ($Y_i(0)$)
\begin{itemize}
\item Fundamental problem of missing data: Individual $i$ cannot have both $Y_i(1)$ and $Y_i(0)$ - at most one of them
\end{itemize}
\end{itemize}
 \end{wideitemize}
\end{frame}

\begin{frame}
\frametitle{Potential vs observed outcome}
\begin{wideitemize}
\item We can bridge the two with this relation
\[
\begin{aligned}
Y_ i &= Y_i(1)W_i + Y_i(0)(1-W_i)\\
&=Y_i(0)+W_i(Y_i(1)-Y_i(0))\\
\end{aligned}
\]
\begin{itemize}
\item We know $W_i$ and $Y_i$ for everyone regardless of treatment assignment
\item We cannot see $Y_i(0)$ for the treated group and $Y_i(1)$ for the untreated group
\end{itemize}
 \end{wideitemize}
\end{frame}

\begin{frame}
\frametitle{Treatment effect}
\begin{wideitemize}
\item If we want to see if the treatment has any effect, we would ideally see
\[
Y_i(1)-Y_i(0)
\]
\item But they cannot be obtained b/c fundamental problem of missing data
\begin{itemize}
\item Alternative is average treatment effect or average treatment effect on the treated
\[
\begin{aligned}
\text{ATE}&=E[Y_i(1)-Y_i(0)]\\
\text{ATT}&=E[Y_i(1)-Y_i(0)|W_i=1]
\end{aligned}
\]
\item We also need assumptions about our treatment: Randomized assignment is one of them
\[
\begin{aligned}
E[Y_i(1)]&= E[Y_i(1)|W_i=1]=E[Y_i(1)|W_i=0] \\
E[Y_i(0)]&= E[Y_i(0)|W_i=1]=E[Y_i(0)|W_i=0] 
\end{aligned}
\]
\end{itemize}
 \end{wideitemize}
\end{frame}

\begin{frame}
\frametitle{Obtaining ATE}
\begin{wideitemize}
\item With this assumption and the definition of potential outcomes framework
\[
\begin{aligned}
E[Y_i|W_i]&=E[Y_i(1)W_i + Y_i(0)(1-W_i)|W_i]\\
&=E[Y_i(1)|W_i]W_i+E[Y_i(0)|W_i](1-W_i)\\
&=E[Y_i(1)]W_i+E[Y_i(0)](1-W_i)\\
\end{aligned}
\]
\begin{itemize}
\item $W_i=1$: Get $E[Y_i(1)]=E[Y_i|W_i=1]$.
\item $W_i=0$: Get $E[Y_i(0)]=E[Y_i|W_i=0]$. 
\end{itemize}
\item Under random assignment, we can identify the average treatment effect as
\[
\text{ATE}=E[Y_i(1)]-E[Y_i(0)] = E[Y_i|W_i=1]-E[Y_i|W_i=0]
\]
\item Econometrically: OLS with $W_i$ as independent variable (Problem set 2!)
 \end{wideitemize}
\end{frame}
%%%%%%%%%%%
\end{document}
